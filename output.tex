% Generated by the Eilenberg-MacLane machine
% Institute of Mathematics, University of Lausanne, Switzerland

\documentclass[12pt,a4paper]{article}
\usepackage{amsfonts}
\usepackage{supertabular}
\begin{document}
\newcommand{\Z}{\mathbb{Z}}
\renewcommand{\thefootnote}{\fnsymbol{footnote}}
\section*{The Eilenberg-MacLane machine
\footnote{\tiny Alain Cl\'ement, Ph.D. Thesis, Institute of Mathematics, University of Lausanne, Switzerland.}}
{\it A C++ program designed to compute the integral homology and cohomology groups of Eilenberg-MacLane spaces.}\\
\subsection*{Homology and cohomology groups of $K(\Z/2^{1},2)$.}\tablehead{\hline%
	$n$ &$H_n(-,\Z)$ &$H^n(-,\Z)$\\%
	\hline &&\\}
\tabletail{\hline%
	\multicolumn{3}{r}{%
	\small\slshape to be continued on the next page}\\}
\tablelasttail{\hline}
\begin{supertabular}{|c|p{5.5cm}|p{5.5cm}|}
$0$%
&$\Z$%
&$\Z$\\

$1$%
&$(0)$%
&$(0)$\\

$2$%
&$\Z/2$%
&$(0)$\\

$3$%
&$(0)$%
&$\Z/2$\\

$4$%
&$\Z/2^{2}$%
&$(0)$\\

$5$%
&$\Z/2$%
&$\Z/2^{2}$\\

$6$%
&$\Z/2$%
&$\Z/2$\\

$7$%
&$\Z/2$%
&$\Z/2$\\

$8$%
&$\Z/2\oplus\Z/2^{3}$%
&$\Z/2$\\

$9$%
&$(\Z/2)^{\oplus2}$%
&$\Z/2\oplus\Z/2^{3}$\\

$10$%
&$(\Z/2)^{\oplus2}$%
&$(\Z/2)^{\oplus2}$\\

$11$%
&$(\Z/2)^{\oplus3}$%
&$(\Z/2)^{\oplus2}$\\

$12$%
&$(\Z/2)^{\oplus2}\oplus\Z/2^{2}$%
&$(\Z/2)^{\oplus3}$\\

$13$%
&$(\Z/2)^{\oplus3}$%
&$(\Z/2)^{\oplus2}\oplus\Z/2^{2}$\\

$14$%
&$(\Z/2)^{\oplus5}$%
&$(\Z/2)^{\oplus3}$\\

$15$%
&$(\Z/2)^{\oplus4}$%
&$(\Z/2)^{\oplus5}$\\

$16$%
&$(\Z/2)^{\oplus4}\oplus\Z/2^{4}$%
&$(\Z/2)^{\oplus4}$\\

$17$%
&$(\Z/2)^{\oplus7}$%
&$(\Z/2)^{\oplus4}\oplus\Z/2^{4}$\\

$18$%
&$(\Z/2)^{\oplus6}$%
&$(\Z/2)^{\oplus7}$\\

$19$%
&$(\Z/2)^{\oplus8}$%
&$(\Z/2)^{\oplus6}$\\

$20$%
&$(\Z/2)^{\oplus8}\oplus\Z/2^{2}$%
&$(\Z/2)^{\oplus8}$\\

$21$%
&$(\Z/2)^{\oplus9}$%
&$(\Z/2)^{\oplus8}\oplus\Z/2^{2}$\\

$22$%
&$(\Z/2)^{\oplus11}$%
&$(\Z/2)^{\oplus9}$\\

$23$%
&$(\Z/2)^{\oplus12}$%
&$(\Z/2)^{\oplus11}$\\

$24$%
&$(\Z/2)^{\oplus12}\oplus\Z/2^{3}$%
&$(\Z/2)^{\oplus12}$\\

$25$%
&$(\Z/2)^{\oplus14}$%
&$(\Z/2)^{\oplus12}\oplus\Z/2^{3}$\\

$26$%
&$(\Z/2)^{\oplus17}$%
&$(\Z/2)^{\oplus14}$\\

$27$%
&$(\Z/2)^{\oplus17}$%
&$(\Z/2)^{\oplus17}$\\

$28$%
&$(\Z/2)^{\oplus18}\oplus\Z/2^{2}$%
&$(\Z/2)^{\oplus17}$\\

$29$%
&$(\Z/2)^{\oplus22}$%
&$(\Z/2)^{\oplus18}\oplus\Z/2^{2}$\\

$30$%
&$(\Z/2)^{\oplus22}$%
&$(\Z/2)^{\oplus22}$\\

$31$%
&$(\Z/2)^{\oplus25}$%
&$(\Z/2)^{\oplus22}$\\

$32$%
&$(\Z/2)^{\oplus27}\oplus\Z/2^{5}$%
&$(\Z/2)^{\oplus25}$\\

$33$%
&$(\Z/2)^{\oplus29}$%
&$(\Z/2)^{\oplus27}\oplus\Z/2^{5}$\\

$34$%
&$(\Z/2)^{\oplus32}$%
&$(\Z/2)^{\oplus29}$\\

$35$%
&$(\Z/2)^{\oplus36}$%
&$(\Z/2)^{\oplus32}$\\

$36$%
&$(\Z/2)^{\oplus36}\oplus\Z/2^{2}$%
&$(\Z/2)^{\oplus36}$\\

$37$%
&$(\Z/2)^{\oplus41}$%
&$(\Z/2)^{\oplus36}\oplus\Z/2^{2}$\\

$38$%
&$(\Z/2)^{\oplus45}$%
&$(\Z/2)^{\oplus41}$\\

$39$%
&$(\Z/2)^{\oplus47}$%
&$(\Z/2)^{\oplus45}$\\

$40$%
&$(\Z/2)^{\oplus50}\oplus\Z/2^{3}$%
&$(\Z/2)^{\oplus47}$\\

$41$%
&$(\Z/2)^{\oplus56}$%
&$(\Z/2)^{\oplus50}\oplus\Z/2^{3}$\\

$42$%
&$(\Z/2)^{\oplus59}$%
&$(\Z/2)^{\oplus56}$\\

$43$%
&$(\Z/2)^{\oplus63}$%
&$(\Z/2)^{\oplus59}$\\

$44$%
&$(\Z/2)^{\oplus69}\oplus\Z/2^{2}$%
&$(\Z/2)^{\oplus63}$\\

$45$%
&$(\Z/2)^{\oplus72}$%
&$(\Z/2)^{\oplus69}\oplus\Z/2^{2}$\\

$46$%
&$(\Z/2)^{\oplus78}$%
&$(\Z/2)^{\oplus72}$\\

$47$%
&$(\Z/2)^{\oplus85}$%
&$(\Z/2)^{\oplus78}$\\

$48$%
&$(\Z/2)^{\oplus87}\oplus\Z/2^{4}$%
&$(\Z/2)^{\oplus85}$\\

$49$%
&$(\Z/2)^{\oplus95}$%
&$(\Z/2)^{\oplus87}\oplus\Z/2^{4}$\\

$50$%
&$(\Z/2)^{\oplus103}$%
&$(\Z/2)^{\oplus95}$\\

&&\\\end{supertabular}
\newpage\subsection*{Generators involved in the calculus.}\tablehead{\hline%
	Degree &Genus &Generator\\%
	\hline &&\\}
\tabletail{\hline%
	\multicolumn{3}{r}{%
	\small\slshape to be continued on the next page}\\}
\tablelasttail{\hline}
\begin{supertabular}{|c|c|p{9cm}|}
$2$
&$1$ &$(\sigma^{2},\sigma\psi_2)$\\
$5$
&$3$ &$(\beta_2\varphi_2\psi_2,\varphi_2\psi_2)$\\
$9$
&$3$ &$(\beta_2\varphi_2\gamma_2\psi_2,\varphi_2\gamma_2\psi_2)$\\
$17$
&$3$ &$(\beta_2\varphi_2\gamma_2^{2}\psi_2,\varphi_2\gamma_2^{2}\psi_2)$\\
$33$
&$3$ &$(\beta_2\varphi_2\gamma_2^{3}\psi_2,\varphi_2\gamma_2^{3}\psi_2)$\\
&&\\\end{supertabular}
\end{document}
